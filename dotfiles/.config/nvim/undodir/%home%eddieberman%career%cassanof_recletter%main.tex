Vim�UnDo�C��ُ��4w�wA���3/@U��:�}W�c�e��_��������Ve�����7Z$\Lambda$CDM cosmology is our current best model for the breakdown and evolution of our universe. The $\Lambda$ term comes from Einstein's field equations \citep{https://doi.org/10.1002/andp.19163540702} and describes the expansion of space time; The CDM stands for cold dark matter, which has been famously measured to be $\sim 5\times$ more abundant than ordinary matter. The $\Lambda$CDM model may strike some as extremely counter intuitive. A priori, we should have no reason to expect that the universe is expanding or that invisible matter is predominant in our universe. This begs the question, what are the $\Lambda$CDM cosmological parameters that have the biggest impact on the structure of the universe? This question has motivated improvements in telescope technology so that we can better measure these parameters \citep{2003SPIE, 2005SPIE, 2012SPIE}. Specifically, the James Webb Space Telescope (JWST) images are reaching new summits of resolution. Furthermore, the JWST images are taken in the infrared range, which allows for us to ``see through'' some of the dust in the way of our observables. In the effort to capitalize on the advancements in hardware, scientists are continually trying to refine their statistical techniques for estimating these cosmological parameters that make up our $\Lambda$CDM model \citep{Corasaniti_2021, PhysRevD.102.023509, 2022PhRvD.105b3520A}. These parameters correspond not only to things like the percentage of dark matter, dark energy, and ordinary matter in the universe, but also more specific things such as the relative flatness of space-time and the Hubble Constant.EThe purpose of my project is to do this parameter estimation for COSMOS-Web: The James Webb Space Telescope Cosmic Origins Survey \citep{Casey_2023}. This gives us the opportunity to verify the values of cosmological parameters using the most sophisticated telescope ever built and potentially find holes in existing physics.*{\center\textbf{Objectives and Methods}\\}�As just mentioned, the objective of this research is to measure the parameters in our $\Lambda$CDM cosmology model using data from the COSMOS-Web survey. This project has two main parts. \begin{enumerate}i\item I need to derive mathematical functions for the posterior distributions of the parameters I am trying to measure. These functions will be constrained by the prior distributions of some variables that are observable. In my case, these observables are galaxy shape, galaxy position, and galaxy red shift. Note, galaxy red shift is just a proxy for distance.�\item Once we have these prior and posterior distributions, we can use the CosmoSIS software \citep{2015A&C....12...45Z} to combine the priors with our data to estimate the desired parameters from the posterior distribution automatically via sampling. \end{enumerate} To accomplish the first part, I can leverage the mathematical maturity obtained during my last project, where I solved a similar constrained optimization problem to accelerate a common parameter estimation task used to calibrate a telescope \citep{berman2023shoptjl}. For the second part of the project, I can leverage the expertise here at Northeastern in using CosmoSIS. In addition to my advisor, I plan on consulting her colleague in Professor Blazek and his group who are very experienced at using the tool.�During the weeks $1$ and $2$ of the project, I will take care of the preliminary work. This entails reading more of the literature, downloading and configuring required software, and setting up documentation. ~Weeks $3-4$ will be spent doing the math to calculate the appropriate posterior distributions I want CosmoSIS to sample from. �Then, weeks $5-6$ will be setting up the more intense calculations on discovery. It is during this week that I will be running CosmoSIS and extracting the necessary data from our JWST data sets in order to do so.�During week $7$ I will conduct an ablation study. This entails tweeking our priors slightly to see how sensitive our model is to tunable hyper parameters. �Once that is done, I will write up these results in week $8$. In the process, I will compare my findings to the results of previous sky surveys such as SDSS, DES, KDS, and Euclid.�I am giving myself two weeks of leeway, to account for any unforeseen delays that I may encounter in this project. If all else goes according to plan, weeks $9$ and $10$ will be spent working with reviewers on any corrections I have to make for the paper.uI have all of the resources I need with access to the Discovery and Candide High Performance Computing superclusters.;{\center\textbf{Outcomes, Evaluation, and Dissemination}\\}(The CosmoSIS software automatically outputs confidence intervals for the learned parameters and will tell you if the new data defies previously observed data with any statistical significance. The null hypothesis suggests that we should expect the new estimates to be consistent with what has been measured in other surveys. This result is still publishable because it stems from a new data set from a state of the art telescope. Specifically, there have been few measurements of the $\sigma_8$ parameter in the red shift range of our data set. I plan on submitting these results for publication in a AAS journal, most likely the Astronomical Journal. I am also planning on speaking at the AAS winter conference session next January, the major conference in Astrophysics that pulls in 2 to 3 thousand people.%{\center\textbf{About the Learner}\\}	I am very determined to obtain a career in academia. Specifically, I want to answer questions about the ``shape'' of data. To that end, my last research project, funded in part by the \textsc{PEAK Ascent Award}, resulted in a first author publication submitted to the Journal of Open Source Software \citep{berman2023shoptjl}. In this project, I leveraged the geometry of a cylinder to accelerate a common parameter estimation task for telescope calibration. A companion paper is in preparation for the Astronomical Journal. The project that I am proposing builds off of my last project in that it involves the same mathematics of constrained optimization and understanding the geometry of data. It also builds off of a PhD level class I took, Riemannian Optimization, as well as several previous research projects. These projects include an internship studying TinyML at the Air Force Research Laboratory, an independent study course on data driven model discovery, and an REU studying control theory for chemical reaction networks.lI am concurrently applying to be nominated for the \textsc{Goldwater Scholarship}, and I plan on applying for the \textsc{NSF GRFP} this fall to support my graduate school endeavors. The \textsc{Summit Award} and other awards just mentioned are natural stepping stones after having obtained the \textsc{Ascent Award} and \textsc{Physics Research Co-op Fellowship}.\newpage*{\center\textbf{Annotated Bibliography}\\}S\cite{2022PhRvD.105b3520A} and \cite{PhysRevD.102.023509} are the Dark Energy Survey Year $1$ and Year $3$ results. In these results, they lay out in great detail their methodologies for calculating the posterior distributions for the parameters they are trying to measure as well as the results from sampling said posterior distributions.;\cite{Corasaniti_2021} does the same kind of measurement as \cite{2022PhRvD.105b3520A} and \cite{PhysRevD.102.023509}. While the uncertainties in \cite{Corasaniti_2021} are less competitive, the observables in this data are galaxy clusters, which is more in line with the what I have available to me in my data set.)\cite{Casey_2023} is an overview paper of the science goals of the COSMOS-Web survey. In this overview paper, doing the type of measurements in the COSMOS field that I am seeking to do is an explicit goal. This paper also underscores my Professor's involvement in the survey as she is a co-author.R\cite{berman2023shoptjl} is the paper that I submitted with my advisor earlier this year. In this paper I solve a constrained optimization problem for a different task: telescope calibration. While this project has a different theme this paper shows that I have an adequate mathematical background to solve problems in a research setting.�\cite{2015A&C....12...45Z} outlines the CosmoSIS software: what it does, how it works, who should use it, and a little on how to use it. In a sense, it acts as an instruction manual for other astronomers looking to do parameter estimation problems.X\cite{https://doi.org/10.1002/andp.19163540702} is a seminal paper from Albert Einstein on his theory of general relativity. Here, Einstein proposes that space and time are actually one space-time. Moreover, he argues that space-time curves around massive objects. His inclusion of the $\Lambda$ term in his famous field equations leaves room for the possibility of an expanding universe, which we have since confirmed via measurement. I included it to show the historical development of our $\Lambda$CDM cosmological model from Einstein's work and underscore the complexity and counter nature of it.7\cite{2003SPIE, 2005SPIE, 2012SPIE} all give technical specifications for the James Webb Space Telescope from its $6$ meter diameter to its $0.3$'' pixel scale. These sources show the 20 year development of the telescope's technology motivated in part by the measurement of $\Lambda$CDM cosmological parameters.8See next page for full citations of sources cited here. 5���7�%5��